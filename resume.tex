\setcounter{page}{5}
\addcontentsline{toc}{chapter}{Abstract}

     \begin{center}
{\large\textbf{Abstract}}
    \end{center}
\vspace*{0.4cm}

The three-dimensional picture of quarks and gluons in the nucleon is set to be 
revealed through deeply virtual Compton scattering (DVCS). With the absence of 
a free neutron target, the deuterium target represents the simplest nucleus to 
be used to probe the internal 3D partonic structure of the quasi-free neutron.  
We propose to measure the beam spin asymmetry (BSA) in incoherent neutron DVCS 
together with the approved BONuS12 experiment, using the same beam time and are 
asking for the electron beam to be highly polarized. The DVCS BSA on a 
(quasi-free) neutron will be measured in a wide range of kinematics by tagging 
the scattered electron and the real photon final state with the spectator 
proton. We will also measure BSA with all final state particles detected; the 
scattered electron, the real photon, the spectator proton, and the struck 
neutron. Both measurements of BSA of neutron DVCS, by tagging the recoil proton 
and in the fully exclusive final state, will help to understand the impact of 
the final state interactions (FSI) and Fermi motion on the incoherent neutron 
DVCS. The proposed measurements are highly complementary to the approved CLAS12 
experiment E12-11-003, which will also measure the quasi-free neutron DVCS by 
detecting the scattered neutron.

\newpage

