%
The three-dimensional picture of quarks and gluons in the 
proton is set to be revealed through Deeply virtual Compton scattering  
while a critically important puzzle in the one-dimensional picture remains, 
namely, the origins of the EMC effect. Incoherent nuclear DVCS, i.e. DVCS on 
a nucleon inside a nucleus, can reveal the 3D partonic structure of 
the \emph{bound nucleon} and shed a new light on the EMC effect.
However, the Fermi motion of the struck nucleon, off-shell effects and 
final-state interactions (FSIs) complicate this parton level interpretation. 
We propose here a measurement of incoherent DVCS with a tagging of the 
recoiling spectator system (nucleus A-1) to systematically control nuclear effects.
Through spectator-tagged DVCS, a fully detected final state presents a 
\emph{unique opportunity}
to systematically study these nuclear effects and cleanly observe possible 
modification of the nucleon's quark distributions.\\

%Probing large initial nucleon momentum (relative to Fermi momentum) provides an 
%interesting opportunity to study mean field and short range effects.The latter 
%corresponds to configuration where the nucleon and spectator are separated by a 
%small distance. On the other hand, probing smaller initial nucleon momentum in 
%$^4$He is more sensitive to the mean field potential and giving a binding 
%energy 20 MeV per nucleon compared to the case of deuterium where it is 2.2 
%MeV. 
%In order to unambiguously identify mean-field modified nucleons, final-state 
%interactions and short-range effects need to be isolated, understood, and, if 
%needed, corrected. This experiment is designed to specifically address issues 
%of final state interactions, Fermi motion and short range correlations for 
%incoherent DVCS.

We propose to measure the DVCS beam-spin asymmetries (BSAs) on $^4$He and deuterium 
targets. The reaction $^4$He$(e,e^{\prime}\gamma\,p\,^3$H$)$ with a fully detected final state 
has the rare ability to simultaneously quantify FSIs, measure initial nucleon momentum,
and provide a sensitive probe to other nuclear effects at the parton level.
The DVCS BSA on a (quasi-free) neutron will be measured by tagging a spectator proton with a 
deuteron target. Similarly, a bound neutron measurement detects a spectator $^3$He off a $^4$He target.
These two observables will allow for a self-contained measurement of the neutron \emph{off-forward EMC Effect}.\\
%, while additionally providing a clean neutron DVCS measurement with a technique 
%similar to the one used in BoNuS experiments.\\
%Taking the ratio of these asymmetries at fixed DVCS kinematics for different 
%Fermi momenta will provide very important information on the modification of 
%the nucleon as function of the N-N separation.

We will also measure the impact of final state interactions on incoherent DVCS 
when the scattered electron, the real photon, and the struck proton are 
detected in the final state. This will help understand the measurements 
performed on helium during the previous CLAS E-08-024 experiment and will allow 
better measurements of the same channel where both statistics and kinematic
coverage are extended. The measurement of neutron DVCS by tagging the recoil 
proton from a deuterium target is highly complementary to the approved 
CLAS12 experiment E12-11-003 which will also measure quasi-free neutron DVCS by 
detecting the scattered neutron.

