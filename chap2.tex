\chapter{Run conditions and Experimental setup}
\label{chap:physics}

\section{Run Group F conditions}

Run Group F (E12-06-113) has been approved to collect 35 PAC days (100\% 
efficiency) of data on deuterium with 11 GeV electron beam and another six days 
on hydrogen. One of the days of hydrogen data taking will be carried out with a 
low energy electron beam of about 2.2 GeV. The 40~cm long target filled with 7 
atm deuterium gas at room temperature and the 200~nA electron beam will yield a 
combined nucleon luminosity of about 2$\times$10$^{34}$cm$^{-2}$s$^{-1}$, about 
a factor of five below the standard CLAS12 nominal luminosity.

For the detection of the recoil protons, Run Group F is going to install a new 
and enlarged radial time projection chamber (RTPC) and target gas cell 
assembly, very similar to the ones used by the BONuS (E03-012) and eg6 
(E08-024) experiments. The RTPC can detect proton recoil momenta down to a 
lower limit of 70 MeV/c while being insensitive to minimum ionizing particles.  
The RTPC will be replacing the central detector's silicon tracker
and barrel micromegas, while keeping an updated design of the forward 
micromegas (FMT). In the updated version of the FMT, only three layers of 
micromegas will be kept to improve the electron's reconstructed vertex 
resolution while reducing the material in the path of the electrons. In the 
following sections we briefly introduce the experimental setup of Run Group F. 

The Run Group F approved measurement (BONuS12)~\cite{bonus12}, as well as the 
proposed measurements here, will use the same trigger as in 
RG-B~\cite{neutronDVCS}, that is an electron in Forward Tracker with road 
matching. 

%%%%%%%%%%%%%%%%%%
\section{The CLAS12 Spectrometer}
%%%%%%%%%%%%%%%%%%
The CLAS12 spectrometer is designed to operate with 11~GeV beam at an 
electron-nucleon luminosity of $\mathcal{L} = 
1\times10^{35}~$cm$^{-2}$s$^{-1}$. The baseline configuration of the CLAS12 
detector consists of the forward detector and the central detector 
packages~\cite{CD} (see Figure~\ref{fig:fd}).
The CLAS12 Central Detector~\cite{CD} is designed to detect various charged 
particles over a wide momentum and angular range. The main detector package 
includes:
\begin{itemize}
 \item Solenoid Magnet: provides a central longitudinal magnetic field up to 
5~Tesla, which serves to curl emitted low energy M{\o}ller electrons and determine 
particle momenta through tracking in the central detector.
 \item Central Tracker: consists of 3 double layers of silicon strips and 6 
    layers of Micromegas. The thickness of a single silicon layer is  
    \SI{320}{\um}.
 \item Central Time-of-Flight (CTOF): an array of scintillator paddles with a 
    cylindrical geometry of radius 26 cm and length 50 cm; the thickness of the 
      detector is 2 cm with designed timing resolution of $\sigma_t = 50$ ps, 
      used to separate pions and protons up to 1.2 GeV/$c$.
\end{itemize}

We will use the forward detector for electron, photon, and neutron detection.  
The central detector's silicon tracker and barrel micromegas will be removed to 
leave room for the BONuS12 RTPC for the detection of the slow recoiling protons 
in deuterium. In addition to the main CTOF of CLAS12 Central detector, we will 
be using the Central Neutron Detector (CND) for the detection of the final 
state recoiling neutrons.  

\begin{figure}
  \begin{center}
    \includegraphics[angle=0, width=0.75\textwidth]{figures/clas12_bonus12.png}
    \includegraphics[angle=0, width=0.75\textwidth,clip, trim = 0mm 10mm 0mm 
     40mm]{figures/bonus12_fmt.png}
     \caption{(Top) The schematic layout of the CLAS12 baseline design with 
     BONuS12 RTPC replacing the silicon tracker and the barrel micromegas.  
     (Bottom) Schematic layout showing BONuS12 RTPC with the modified design of 
     the forward micromegas.}
    \label{fig:fd}
  \end{center}
\end{figure}


The scattered electron, photon, and some neutrons will be detected in the 
forward detector which consists of the High Threshold Cherenkov Counters 
(HTCC), Drift Chambers (DC), the Low Threshold Cherenkov Counters (LTCC), the 
Time-of-Flight scintillators (TOF), the Forward Calorimeter and the Preshower 
Calorimeter. The charged particle identification in the forward detector is 
achieved by utilizing the combination of the HTCC and TOF arrays with the 
tracking information from the Drift Chambers. The HTCC together with the 
Forward Calorimeter and the Preshower Calorimeter will provide a pion rejection 
factor of more than 2000 up to a momentum of 4.9~GeV/c, and a rejection factor 
of 100 above 4.9 GeV/c. The photons and the neutrons are detected using the 
calorimeters. As will be showing later on, the majority of the final state 
recoiling neutrons will be detected using the sub-detectors of CLAS12 Central 
Detector, in particular CTOF and CND. 

\section{BONuS12 RTPC} 

\begin{figure}
  \begin{center}
    \includegraphics[angle=0, width=0.45\textwidth, clip,trim=50mm 10mm 80mm 
     0mm]{figures/Bonus12_cad.png}
    \includegraphics[angle=0, width=0.45\textwidth,clip,trim=0mm 10mm 20mm 0mm 
     ]{figures/NKsBXp.png}
     \caption{(Left) Schematic layout showing BONuS12 RTPC showing the readout 
     padboard and few adaptor boards in addition to the gas lines. (Right) 
     Schematic drawing of the CLAS RTPC in a plane perpendicular to the beam 
     direction. See text for description of the elements.}
    \label{fig:bonus12}
  \end{center}
\end{figure}

The new CLAS12 RTPC (BONuS12) is 400~mm long cylinder of 160~mm diameter. The 
electric field is directed perpendicularly to the beam direction, such that 
drifting electrons are pushed away from the beam line.  These electrons are 
amplified by three layers of cylindrical gas electron multipliers (GEM) and 
detected by the readout system on the external shell of the detector as 
illustrated in Figure~\ref{fig:bonus12}. The BONuS12 RTPC covers almost 100\% 
of the azimuthal angle range.

We detail here the different regions shown in Figure~\ref{fig:bonus12} starting 
from the beam line towards larger radius:\\
\begin{itemize}
  \item The 7~atm Deuterium gas target extends along the beamline forming the 
     detector central axis. It is a 6~mm diameter Kapton straw with a 50~$\mu$m 
      wall of 492~mm length such that its entrance and exit 15~$\mu$m aluminum 
      windows are placed outside of the detector volume.  The detector and the 
      target are placed in the center of the solenoid.
   \item The first gas gap covers the radial range from 3~mm to 20~mm. It is 
      filled with $^{4}$He gas at 1~atm to minimize secondary interactions from
      M\o{}ller electrons scattered by the beam. This region is surrounded by a 
      4~$\mu$m thick window made of grounded aluminized Mylar.
   \item The second gas gap region extends between 20~mm and 30~mm and is 
      filled with the gas mixture of 80$\%$ $^{4}$He and 20$\%$ CO$_2$. This 
      region is surrounded by a 4~$\mu$m thick window made of aluminized Mylar 
      set at $-4260$~V to serve as the cathode.
   \item The drift region is filled with the same $^4$He-CO$_2$ gas mixture and 
      extends from the cathode to the first GEM, 70~mm away from the beam axis.  
      The electric field in this region is perpendicular to the beam and 
      averages around 550~V/cm.
   \item The electron amplification system is composed of three GEMs located at 
      radii of 70, 73 and 76~mm. The first GEM layer is set to $\Delta 
      V=-1620$~V relative to the ground and then each subsequent layer is set 
      to a lower voltage relative to the previous to obtain a strong 
      ($\sim$1600~V/cm) electric field between the GEM foils. A 275~V bias is 
      applied across each GEM for amplification.
   \item The readout board has an internal radius of 79~mm and collects charges 
      after they have been multiplied by the GEMs. Adaptor circuit boards are 
      plugged directly on its outer side and transmit the signal to the Hitachi 
      cables connected to the BMT DAQ electronics.
\end{itemize}



\section{Beam Polarization}
For our proposed measurements we ask for the electron beam to be highly 
polarized, i.e., 85\% longitudinally polarized beam, which is the average 
achieved polarization of the beam during Run Group A and Run Group B using the 
upgraded 12 GeV experimental setup. Regarding measuring the beam polarization, 
no additional commissioning time is required. The spokespersons of Run Group F 
\cite{skuhn} has accepted to schedule time to perform M{\o}ller runs once or 
twice a week, which is enough for the proposed measurements.    

