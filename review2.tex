\chapter{2nd round of comments}
{\bf Authors' response to the second round comments on CLAS Run-Group F 
additional proposal: Neutron DVCS Measurements with BONuS12 in CLAS12 \\ by 
M.Hattawy et al.}
\author{Angela Biselli, Francesco Boss\`u (chair), Rafayel Paremuzyan\\CLAS 
Analysis Review Committee}

{\center \bf CLAS Analysis Review Committee:\\Angela Biselli,\\ Francesco 
Boss\`u (chair),\\ Rafayel Paremuzyan\\}


\def \rarr {\ensuremath{\rightarrow}}
~\\
~\\
~\\

{\bf N.B. the answers from the authors are in blue.} \\ 
 
 \section*{General comments}
 \begin{itemize}
  \item Equations 1.14. $\alpha_s$ has a very distinctive meaning in QCD, so 
     the choice of calling this variable like the strong coupling constant is 
       very unfortunate. Please, rename it.
  
  \textcolor{blue}{Response: Agree. The ``s'' has been change to ``sp'' in the 
       notation of $\alpha$. }


\item Naming ``Tagged neutron DVCS'' the events where the neutron is missed is 
   misleading, as the word \textit{tagged} suggests that you have identified 
       the particle in question. So, we suggest to name it ``proton-tagged'' 
       nDVCS.

 
  \textcolor{blue}{Response: Has been changed everywhere in the proposal.}
 
 \end{itemize}

 
 
 \section*{Comments/questions to  the authors' comments}
 
  
 \begin{itemize}
  {\color{red} \item  -Our first intention here is to extract results in order 
     to validate the available FSI
  \item  As explained previously, we intend to extact results to be compared to 
     the theoretical models. For instance, we will compare the 
       theta-distribution of spectator protons for fully identified DVCS events 
       for 2 different momentum bins - e.g. 70-100 MeV/c and 100-150 MeV/c 
       (there are many papers showing the effect of FSI in the original
 BONuS12 and DEEPS data in these momentum bins - e.g. https://journals.aps.
 org/prc/pdf/10.1103/PhysRevC.93.055205).
 models.} \newline
 
 It would be much clearer and understandable for us, if you could provide 
       predictions of available models (at least one or few of them, if there 
       are existing models), and expected uncertainties of the data in order to 
       see the constraining power of the data, However, if there are not 
       available (mature) models for deep exclusive reactions, it is still very 
       valuable to do this measurements, since it can trigger theorists to try 
       to make models. \\
  Please clarify this.

  
  \textcolor{blue}{Response: }
  
  \textcolor{red}{
  \item   The goal of the proposed measurement is not to ``improve'' on RG-B, 
     but to provide an independent test. Neutron DVCS has been only measured in 
       few instances to this point and the evaluation of systematic errors 
       without data is by nature a complex matter. The main point is that this 
       experiment will have very different systematics and can check some 
       assumptions used to interpret RG-B measurement.
  }
  \newline
  
  It is indeed clear that the intention is not to combine the data with RG-B.  
       Nevertheless, in Section 1.2, it is stated that more data ``are always 
       important'' and given the different systematics the two methods ``are 
       complementary''. So, given the quoted 20\% systematic uncertainty of 
       this proposal, one may wonder about the impact of this measurement in 
       the understanding the impact of Fermi motion and FSI on the RG-B data.
  We suggest maybe to put more emphasis on the fact that it will give 
       indications and may indicate if future experiments (or more data) will 
       be be desirable.
  
  \textcolor{blue}{Response: }
  
       
 \textcolor{red}{
 \item As stated above, we do not believe this run group proposal review is the 
    right place to discuss the specifics of choices made previously for the run 
       conditions of the Bonus12 proposal. However, the lowest p momentum we 
       can reconstruct is not much lower with 4 T than with 5 T, but the 
       background increases and the momentum resolution of the RTPC gets worse}
\newline

We understand that the point of this RG addendum is not to discuss again about 
       the experimental conditions, but we think that giving an estimate of 
       what is lost with 5T is important. You could show the spectator proton 
       momentum before the BONuS12 acceptance cuts. Of course, then, the fact 
       that the RTPC will get too much background is limiting the lower 
       momentum at 70 MeV/c.

  
       \textcolor{blue}{Response: }

\textcolor{red}{
\item Indeed some portion
of the phase space is cut by the absence of the FT. We feel that it is not a critical issue
for comparisons with E12-11-003, we can do it only in part of the phase space and still
obtain a meaningful comparison. On the other side, including the FT would increase
the background in BONuS12 and cause changes on the beam line that we did not want
to ask from the run group as it would imply to reassess things validated by their ERR
}
 \newline

It is not clear why including the FT would increase the background in BONuS12.  
Could you explain?
  
  \textcolor{blue}{Response: }
 



\textcolor{red}{
 \item Yes, we will use the same trigger as in RG-B and BONUS12, that is an 
    electron in Forward Tracker with road matching  }
 \newline
We do not see the discussion on the trigger used in the new draft. You should 
add it.


  \textcolor{blue}{Response: Has been added under section 2.1 Run Group F 
  conditions.}


  \textcolor{red}{
  \item Systematics. $\pi^0$.  }
  \newline
 
 Generally, non-exclusive $\pi^0$ means SIDIS, i.e. a reaction like 
 $e+p\rightarrow e+\pi^0+X$. Here instead you discuss \textbf{exclusive} 
 $\pi^0$ where one of the two photons is missed. Please, change the 
 terminology. \\ Then, what about SIDIS events?
  
  
  \textcolor{blue}{Response: Thanks. It has been changed. Regarding the SIDIS 
  condtribution, our selection criteria of applying the exclusivity cuts cleans 
  most of these events based on our previous work during eg6 analysis. Same 
  assumption has been considered in the RG-B nDVCS measurment.}
  
 \end{itemize}
