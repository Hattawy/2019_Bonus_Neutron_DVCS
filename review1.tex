\chapter{First Round Comments}

{\bf Authors' response to the first round comments on CLAS Run-Group F 
additional proposal: Neutron DVCS Measurements with BONuS12 in CLAS12}\\

{\center \bf CLAS Analysis Review Committee:\\Angela Biselli,\\ Francesco 
Boss\`u (chair),\\ Rafayel Paremuzyan\\}

\def \rarr {\ensuremath{\rightarrow}}
~\\
~\\
~\\

{\bf N.B. Comments from the reviewer are in normal font, answers from the authors are in red.} \\

The authors are proposing an addition to RG-F to study the deeply virtual 
Compton scattering on neutrons (nDVCS) using a deuterium gaseous target. The 
experimental setup of RG-F, and in particular the BONuS12 radial TPC, is 
particularly well suited for the detection of the low-momentum spectator proton 
giving the possibility to study the impact of Fermi motion and final state 
interactions on nDVCS.
 
This experiment is planning to use tagged nDVCS ($\gamma^{*} + D \rarr \gamma + 
p + (n)$), and fully exclusive nDVCS ($\gamma^{*} + D \rarr \gamma + p + n$) 
events. Although a large difference in the expected statistics between the two 
channels, their comparison will allow one to better understand systematic 
uncertainties in the tagged nDVCS analysis. In addition, these proposed 
measurements will complement and possibly help in the understanding of the 
impact of Fermi motion and FSI on the analysis results of RG-B.
 
This proposal demands only one major modification to the foreseen RG-F data 
taking plans: a highly polarized electron beam. The proponents state in this 
proposal that the RG-F spokespersons already agreed to allocate the necessary 
time to perform polarization measurements, therefore this request represents a 
beneficial addition to the RG-F program, because it will enable not only the 
analysis of this particular channel but possibly many other analyses.
  
The current draft is well documented, but we suggest the proponents to better 
highlight the importance of the nDVCS measurements and the complementarity with 
respect to RG-B.
The following comments and questions aim at helping to clarify and improve the 
proposal. Detailed comments and text corrections in the following section.
 
 
 \section*{General comments}
 
  \begin{itemize}
  
\item The authors should further develop the significance of the FSI and Fermi 
   motion. What is the constraining power of the expected results over these 
        effects?

{\color{red} RD: we should add the results from the incoherent in the chapter 1 
        and then make a global statement as requested.\\
        We have  have added it under section 1.3.2 with a brief description.  
        Please cross check.}

\item There should be more details  on how this  proposal is a ``companion'' of 
   the RG-B proposal. What does this analysis intend to accomplish differently 
        from RG-B?  \\ E12-11-003 quote total $8\%$ systematic uncertainties, 
        while in this proposal estimated uncertainties are $20\%$. Do you have 
        a quantitative evaluation on how much improvement this experiment is 
        going to impose over the 12-11-003 experiment?

 {\color{red} In our opinion, the discrepancies are mostly linked to the 
        optimism of the authors of the proposals at this point. Neutron DVCS 
        has been only measured in few instances to this point and the 
        evaluation of systematic error bars without data is by nature a complex 
        matter. In any case, we expect the final results of both experiments to 
        yield rather similar systematic error bars.}
 
 \item The experiment is planning to run with 5 times less luminosity than the 
    CLAS12 designed luminosity, and hence these runs are expected to have 
        smaller background and, so, larger tracking efficiency in the forward 
        detector. But during the RG-A and RG-B data taking, the beam current 
        was limited to about 2/3 of the nominal request due to occupancy in the 
        tracking detector higher than expected. Did you consider how much would 
        the loss of statistics impact the expected results? 
 
{\color{red} We do not think it is the place of a run group proposal to 
        reassess the feasibility of a proposal running conditions. This is the 
        job of the ERR organized by JLab, which has been passed for the run 
        group discussed here 
        (\url{https://userweb.jlab.org/~kuhn/BONuS12/Documents/ERR/BONUS_ERR_Report_Response.pdf}).  
        Nonetheless, we can say here that the limit in luminosity of the run 
        group is driven by the RTPC performance and cannot be directly related 
        to passed performances of the RG-A or B.}

 \item Do Moller electrons represent an issue for the tracking performance? 

 {\color{red} Again, since we run with 30\% of RG-A/B luminosity, the Moller 
        backgrounds are much reduced in RG-F.}

 \item It is not explained why the solenoid will be set at 5 Tesla. Since this 
    proposal aims at measuring the low-momentum spectator proton, one would 
        guess that a lower magnetic field would allow reaching lower momenta.  
        Was this considered?
 
{\color{red} As stated above, we do not believe this run group proposal review 
        is the right place to discuss the specifics of choices made previously 
        for the run conditions of the Bonus12 proposal. However, the lowest p 
        momentum we can reconstruct is not much lower with 4~T than with 5~T, 
        but the background increases and the momentum resolution of the RTPC 
        gets worse.}


\item What is the recoil proton momentum resolution? There is no discussion on 
   the tracking of such low energy protons in the solenoid. How big is the 
   impact of the proton reconstruction resolution on the final results on FSI 
   and Fermi motion?\\In the text there is also a remark regarding the fact 
   that the reconstruction with BONuS12 is not yet finalized. We suggest to 
   smoothen the fact that the tracking is not yet included in the CLAS12 
   framework and to give an estimate on when it will be.
 
{\color{red} As stated previously, this is a run group addition proposal using 
the same experimental setup of the approved BONuS12 proposal. The observed 
track reconstructed momentum resolution is better than 10\% at 100 MeV/c 
momentum. The resolution of the RTPC has very little impact on the FSI results 
because cuts are very wide. For Fermi motion, a better resolution is always an 
improvement, but with varied effectiveness. At some point resolution 
improvement has very little impact on errors dominated by other factors.  
Regarding the BONuS12 RTPC track reconstruction, it is finalized on a 
standalone version of COATJAVA and got accepted by the CLAS12 software group, 
while we are going through a cleaning phase to include it with the official 
CLAS12 framework.}

\item In the experimental setup, there is no mention of the Forward Tagger.  
   Did you consider including it? Given that RG-B is running with FT, what will 
   be the impact on the comparison between the two experiments of not including 
   the FT?
{\color{red} Indeed some portion of the phase space is cut by the absence of 
the FT. We feel that it is not a critical issue for comparisons with 
E12-11-003, we can do it only in part of the phase space and still obtain
a meaningful comparison. On the other side, including the FT would increase the 
background in BONuS12 and cause changes on the beam line that we did not want 
to ask from the run group as it would imply to reassess things validated by 
their ERR.}

\item There is no mention of trigger requirements for this measurement. Will 
   you use the same trigger as the BONuS12? Is it just an electron trigger with 
   certain electron momentum range? Please specify.
  
{\color{red} Yes, we will use the same trigger as in RG-B and BONUS12, that is 
an electron in Forward Tracker with road matching.}

\item Systematic uncertainties.  \begin{itemize}
\item  It is written that ``normalization and efficiencies [...] cancel out in 
   the asymmetry ratio''. This is true only if the data taking conditions do 
      not change in time, i.e. all the runs considered have very similar 
      conditions.\\
{\color{red}Since the asymmetry is with respect to the beam helicity, which 
      changes rapidly, this will not be a conceren. Further more, we will 
      select the good runs list.  To this aim, in such analysis we monitor the 
      ratio of the number of the good tracks reconstructed in the RTPC to 
      number of the detected good electrons in CLAS as a function of run 
      number. More details can be found in reference ~\cite{eg6_note}, section 
      4.1.1 and 4.2.1. }

 \item There is a mention of non exclusive $\pi^0$ background. What do you mean 
    by ``non exclusive''? What about the exclusive production of $\pi^0$ with 
      either a missing photon or merged photons? Since this is the major 
      background contribution for the e-p data, could you elaborate more on 
      this source of background?

      {\color{red} The non-exclusive $\pi^0$ events here stands for the 
      production where one of the photons from the $\pi^0$ decay escapes 
      detection. From similar analysis on incoherent proton DVCS during 6 GeV 
      era, the $\pi^0$ contamination was estimated and subtracted using 
      detector simulation and experimental data. From simulation, we calculated 
      the ratio ($R = N^{1\gamma}_{sim}/N^{2\gamma}_{sim}$) of the number of 
      $\pi^0$ events that were wrongly identified as exclusive $ep\rightarrow 
      e'p'\gamma$ events ($N^{1\gamma}_{sim}$) to the number of events 
      correctly identified as exclusive $ep\rightarrow e'p'\pi^0$ 
      ($N^{2\gamma}_{sim}$).  Then in each kinematical bin and for each 
      beam-helicity state, the $\pi^0$-subtracted experimental DVCS events were 
      calculated as $N = N^{ep\rightarrow e'p'\gamma}_{exp}- R~N^{ep\rightarrow 
      e'p'\pi^0}_{exp}$, where $N^{ep\rightarrow e'p'\gamma}_{exp}$ 
      ($N^{ep\rightarrow e'p'\pi^0}_{exp}$) is the number of the experimentally 
      identified $ep\rightarrow e'p'\gamma$ ($ep\rightarrow e'p'\pi^0$) events.  
      Depending on the kinematics, we subtracted between 8 and 10\% of the data 
      due to the $\pi^0$ contamination. More details can be found at  
      ~\cite{eg6_note}. 

}

   
   \item Last paragraph of page 25. Varying the exclusivity cuts means that you 
      change the amount of signal that you accept, therefore the obtained 
      uncertainty also contains a statistical part. Was this taken into 
      account?


      {\color{red} The effect of statistics is actually very small. Indeed, a 
      major part of the two samples are common and therefore do not vary like 
      independent events (which is assumed for statistical error bar 
      evaluation).  Moreover, tightening the cuts removes portions of good and 
      bad events, which also affects the corrections applied to remove the 
      backgrounds. At the end, we use values from the CLAS eg6 experience since 
      it is the closest in nature to what is proposed here. There, the study 
      was performed fully by re-evluating all the corrections for each cut 
      choices.}

 \item Fit sensitivity. What function was used? How this dependence of the 
    binning was evaluated? Did you change the functional form?

 {\color{red} Same as above, we did not perform this study here, but reused the 
      results of the eg6 analysis. More details can be found at  
      ~\cite{eg6_note}.  }

  \end{itemize}
  
  \end{itemize}
 

\section*{Detailed comments} 


 \begin{itemize}
 
  \item Introduction. Line 5. Change ``motion of partons'' in ``longitudinal motion of partons''.

     {\color{red} Done.}
  
  \item Chapter 1. Avoid technical jargon or it needs to be introduced.
       
     {\color{red} Text has been cleaned.}
  
  \item Page 9, line 9. ``proposal is enhance'' \rarr ``proposal is to enhance''.
     
     {\color{red} Done.}
    
  \item  Page 9 $2^{nd}$ paragraph, A question on the following sentence  
     ``However, the impact of the uncertain initial state and the final state 
       interactions on the integration of these data in global fits remain 
       unclear.'' \\
  - How this proposal is going to address/answer to this question. See General comments.
     {\color{red} Answered in details previously.}

  \item Page 9,  4th line from the bottom. ``systematic errors'' \rarr ``systematic uncertainties''.
     {\color{red} Done.}
  
  \item Page 10, line 1. ``plagued'' \rarr ``affected''.
     
     {\color{red} Done.}
  
  \item Page 10, line 6. ``... the sin component of the beam spin asymmetry''.  
     The BSA is not yet introduced, therefore is jargon to discuss the sin 
       component.
     
       {\color{red} Text has been cleaned.}
  
  \item Page 10. Eq 1.1 contains Compton form factors that aren't mentioned 
     yet. Again, a non super expert reader may be lost.
     
       {\color{red} Text has been cleaned.}
  
%   \item Page 10. Eq 1.3. Is $d\varphi$ needed?
  
  \item Page 12 Formulas (1.7), (1.8) and (1.9) \\
  -Typo: $cos$ with $s_{n}$ coefficients should be $sin$.
  
     {\color{red} Corrected.}
  
  \item Page 12 $1^{st}$ line:  ``are called Fourier coefficients and they 
     depend...'' \\
  might be better to change ``are called Fourier coefficients. They depend...''
     
       {\color{red} Done.}
  
  \item Page 12. After Eq 1.12, expand on why you want to measure the BSA.
       
     {\color{red} The following three sections in the updated version are 
       explaining why we are interested in measuring the BSA.}
  
  \item Page 13. ``... uncertainties in detection''. Unclear, please rephrase.
     
     {\color{red} cleaned.}
  
  \item Page 13. Line 10. ``... published data is ...'' \rarr ``published data are...''
     
     {\color{red} Can be either. But will go with your suggestion. Done.}
  
  \item Page 14. Line 1. ``... helping with nuclear effects'' \rarr ``...  
     helping in the understanding of the nuclear effects''.
     
       {\color{red} Done.}
  
  \item Page 14 $3^{rd}$ paragraph: ``we see calculation''\\
  - A typo ``we see calculation...''
     
     {\color{red} Done.}
  
  \item Page 14 $3^{rd}$ paragraph: ``At low recoil momentum and
backwards spectator angle, the FSIs are negligible, whereas at high momenta perpendicular
to the momentum transfer, FSIs are maximized'' \\ 
- This doesn't seem to be consistent with Fig1.5 that you are quoting. It seems 
  labels in the left plot of Fig. 1.5 are wrong. Probably it needs to be 
  changed: $0^{\circ}\rightarrow 180^{\circ}$, $180^{\circ}\rightarrow 
  90^{\circ}$, $90^{\circ}\rightarrow 0^{\circ}$.\\
     {\color{red} 
     - The original figure has been quoted from reference 
       arXiv:nucl-th/0307052v2  24 Jul 2003. We agree with you, that it seems 
       the paper itself had this typo. The proposal has been updated with an 
       edited version of this figure. \\
     }



- PWIA not defined.\\
     {\color{red} - PWIA got defined. }

  
\item What is your intention about large and small FSI regions. 
  - Fig. 3.10 shows bins with different momentum and angles, but it is not 
    clear what you are planning to do in order to understand better FSI and 
    initial motions of nucleons in the deuterium (or nuclei).\\
     {\color{red} -Our first intention here is to extract results in order to 
     validate the available FSI models. }
  
  \item Page 16 $2^{nd}$ paragraph: ``and eg6 (E08-024)experiments'' \\
  - Add space between (E08-024) and experiments.\\
     {\color{red} Done.}
  
 \item Page 17 $3^{rd}$ paragraph ``The charged particle identification in the 
    forward detector is achieved by utilizing the combination of the HTCC, LTCC 
    and TOF arrays with the tracking information from the Drift Chambers.'' \\
- LTCC is present only in four sectors, and a proper gas mixture is only in one 
  sector. Most of charged particle identification doesn't use LTCC. You might 
  skip LTCC, or otherwise if for completeness you want to include it, then you 
  might want to add RICH as well.\\
     {\color{red} Skipped as the fact that BONuS does not request LTCC nor does 
     this proposal.}


\item In the summary page, ``\textit{ The first channel will enrich our 
   knowledge about the partonic structure of the quasi-free neutrons, while the 
   fully exclusive neutron DVCS measurement will be a golden data set to 
   understand the Fermi motion and final state interaction effects on the 
   measured DVCS beam-spin asymmetries.''}\\
- It was not clear how exactly you are going to learn more about FSI and Fermi 
  motions with proposed measurements. See also the General remarks.\\

     {\color{red} As explained previously, we intend to extact results to be 
     compared to the theoretical models. For instance, we will compare the 
     theta-distribution of spectator protons for fully identified DVCS events 
     for 2 different momentum bins - e.g. 70-100 MeV/c and 100-150 MeV/c (there 
     are many papers showing the effect of FSI in the original BONuS12 and 
     DEEPS data in these momentum bins - e.g.  
     \url{https://journals.aps.org/prc/pdf/10.1103/PhysRevC.93.055205}).}

\end{itemize}
 


