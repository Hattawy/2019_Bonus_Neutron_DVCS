\chapter*{Summary \markboth{\bf Summary}{}}
\label{chap:conclusion}
\addcontentsline{toc}{chapter}{Summary}

In summary, polarizing the electron beam during the approved E12-06-113 
experiment (BONuS12) will allow us to investigate in a unique way many aspects 
of QCD within the GPD framework. The approved E12-11-003 experiment,
``Deeply Virtual Compton Scattering on the Neutron with
CLAS12 at 11 GeV'' is set to measure the n-DVCS beam-spin asymmetry by
directly detecting the struck neutron in the reaction $\gamma^{*}+d\rightarrow 
n+\gamma+(p)$. In contrast, we intend to measure the neutron DVCS beam-spin 
asymmetry by tagging the spectator slow-recoiling proton in addition to 
measuring the fully exclusive neutron DVCS channel. The first channel will 
enrich our knowledge about the partonic structure of the quasi-free neutrons, 
while the fully exclusive neutron DVCS measurement will be a golden data set to 
understand the Fermi motion and final state interaction effects on the measured 
DVCS beam-spin asymmetries. While this proposal is focusing only on the neutron 
DVCS measurements, highly polarizing the beam during Run Group F will be giving 
us a golden chance to measure additional physics topics, increasing the physics 
outcome of the approved beam time and advancing our understanding on many 
aspects of QCD.     

