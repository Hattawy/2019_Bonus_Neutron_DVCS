
%\documentclass[a4paper,11pt,twoside]{ThesisStyle}
\documentclass[a4paper,11pt,twoside]{article}

\title{Third round comments from the review committee for "Validation of 
neutrino energy estimation using electron scattering data' by Lawrence 
Weinstein et al." \\
 (Hovanes, Natasha, and Mohammad)}


\date{\today}
\include{formatAndDefs}

\begin{document}

\maketitle

\section{Major comments}

\begin{enumerate}
   \item line 605: What about the new momentum corrections you extracted 
      recently? same question would be addressed in section 3.5 

   \item line 690: please double check what each color distribution stands for?  
      you say that the green stands for $\Delta t_{CC}$ and angle between CC 
      and SC hits, while purple line stands for an additional cut on $\Delta 
      t_{CC}$? what is the difference between the two cuts here? the 
      distributions do not prove what you mention here.  

   \item Section 3.2: Proton identification. Why you do not cut the low momenta 
      protons? In line 912, you say that below 300 MeV/c, the CLAS efficiency 
      for proton detection is not know well? Then why you keep all these low 
      momenta protons in your analysis? This applies everywhere for all your 
      analysis and all beam energies.

   \item line 829: What do you show here? the calculated mass from the 
      reconstructed p and theta? not clear here? what about the 1.1 GeV data?

   \item line 831: please correct which figure stands for what?

   \item Figure 45: the cuts do not represent 2$\sigma$ here? They are 
      arbitrary cuts?

   \item Please move section 3.7: Electron fiducial cuts to before the physics 
      and energy cuts. i.e., at the beginning of the electron identification.

   \item section 3.12: please move this before any physics cut you mention 
      previously. You keep saying vertex cut and fiducial cuts and at the end 
      you explain what are they. Please move them to before any physics cut.   

\end{enumerate}

\section{Minor comments}

\begin{enumerate}
   \item line 16: continue your idea " where the nucleons are undergo Fermi 
      motions, i.e., not totally at rest" or something similar. 

   \item line 463: E$_i$ energy -> E$_i$ is the energy.

   \item line 600: The run and analysis conditions were standard". What do you 
      want to say here? not clear.

   \item line 602: would be nice to at least briefly mention what has been 
      published, which physics channels?

   \item line 604: was also approved "as a" CLAS ... Add "as a"

   \item Table.2: add a line at the bottom of the table.

   \item line 670: are shown in 11 -> are shown in Fig.~11

   \item line 742: add space Fig.~23

   \item line 810 and line 811: flip figures 33 and 34 or flip pi- and pi+.  

   \item lines 851 and 855: add spaces between Fig. and the number.

   \item line 863: flip the figures -> 47, 48, and 49.

   \item "have have selected" -> "have selected". "as the below" -> "as below".

   \item line 920: analysis analysis -> analysis.

   \item line 969: (16)-> Eq.~16.

   \item line 969: what about $p_3$.

\end{enumerate}







\end{document}
