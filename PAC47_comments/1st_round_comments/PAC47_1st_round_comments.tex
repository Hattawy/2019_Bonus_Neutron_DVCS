
%\documentclass[a4paper,11pt,twoside]{ThesisStyle}
\documentclass[a4paper,11pt,twoside]{article}

\title{First round comments from Markus Diehl on the proposal ``Neutron DVCS 
Measurements with BONuS12 in CLAS12''}


\date{\today}
\usepackage{amsmath,amssymb}             % AMS Math
\usepackage[utf8]{inputenc}
\usepackage[OT1]{fontenc}
\usepackage[left=2.5cm,right=2.5cm,top=2cm,bottom=2cm,includefoot,includehead,headheight=13.6pt]{geometry}
\usepackage{setspace}
\usepackage{lineno}
\usepackage{footmisc}
\usepackage{indentfirst}
\usepackage{siunitx}
\usepackage{lmodern}
\usepackage{bm}
\usepackage{float}% If comment this, figure moves to Page 2
%\extrafloats{100}
\usepackage{tabu}
\usepackage{multirow}
\usepackage{cite}

% My pdf code
\usepackage{graphicx,type1cm,eso-pic,color}

% Links in pdf
\usepackage{color}
\definecolor{linkcol}{rgb}{0,0,0.4} 
\definecolor{citecol}{rgb}{0.5,0,0} 

\usepackage{array}
\newcolumntype{L}[1]{>{\raggedright\let\newline\\\arraybackslash\hspace{0pt}}m{#1}}
\newcolumntype{C}[1]{>{\centering\let\newline\\\arraybackslash\hspace{0pt}}m{#1}}
\newcolumntype{R}[1]{>{\raggedleft\let\newline\\\arraybackslash\hspace{0pt}}m{#1}}

% Change this to change the informations included in the pdf file

% See hyperref documentation for information on those parameters
%\usepackage{hyperref}
%\hypersetup
%{
%bookmarksopen=true,
%pdftitle="ALERT Run Group Proposal",
%pdfauthor="Rapha\"el Dupr\'e", 
%pdfsubject="ALERT Run Group Proposal", %subject of the document
%%pdftoolbar=false, % toolbar hidden
%pdfmenubar=true, %menubar shown
%pdfhighlight=/O, %effect of clicking on a link
%colorlinks=true, %couleurs sur les liens hypertextes
%pdfpagemode=UseNone, %aucun mode de page
%pdfpagelayout=SinglePage, %ouverture en simple page
%pdffitwindow=true, %pages ouvertes entierement dans toute la fenetre
%linkcolor=linkcol, %couleur des liens hypertextes internes
%citecolor=citecol, %couleur des liens pour les citations
%urlcolor=linkcol %couleur des liens pour les url
%}

% definitions.
% -------------------

\setcounter{secnumdepth}{3}
\setcounter{tocdepth}{2}

% Some useful commands and shortcut for maths:  partial derivative and stuff
\newcommand{\xbp}{$x_{Bj}$}
\newcommand{\xb}{$x_{Bj}~$}
\newcommand{\ptp}{$p_\perp^2$}
\newcommand{\pt}{$p_\perp^2~$}
\newcommand{\dptp}{$\Delta \langle p_\perp^2 \rangle$}
\newcommand{\dpt}{$\Delta \langle p_\perp^2 \rangle ~$}

\newcommand{\vect}[1]{\boldsymbol{#1}}

\brokenpenalty10000\relax

\newcommand{\pd}[2]{\frac{\partial #1}{\partial #2}}
\def\abs{\operatorname{abs}}
\def\argmax{\operatornamewithlimits{arg\,max}}
\def\argmin{\operatornamewithlimits{arg\,min}}
\def\diag{\operatorname{Diag}}
\newcommand{\eqRef}[1]{(\ref{#1})}

\usepackage{rotating}                    % Sideways of figures & tables
%\usepackage{bibunits}
%\usepackage[sectionbib]{chapterbib}          % Cross-reference package (Natural BiB)
%\usepackage{natbib}                  % Put References at the end of each chapter
                                         % Do not put 'sectionbib' option here.
                                         % Sectionbib option in 'natbib' will do.
\usepackage{fancyhdr}                    % Fancy Header and Footer

% \usepackage{txfonts}                     % Public Times New Roman text & math font
  
%%% Fancy Header %%%%%%%%%%%%%%%%%%%%%%%%%%%%%%%%%%%%%%%%%%%%%%%%%%%%%%%%%%%%%%%%%%
% Fancy Header Style Options

\pagestyle{fancy}                       % Sets fancy header and footer
\fancyfoot{}                            % Delete current footer settings

\fancyhead[LE,RO]{\bfseries\thepage}    % Page number (boldface) in left on even
% pages and right on odd pages
\fancyhead[RE]{\bfseries\nouppercase{\leftmark}}      % Chapter in the right on even pages
\fancyhead[LO]{\bfseries\nouppercase{\rightmark}}     % Section in the left on odd pages

\let\headruleORIG\headrule
\renewcommand{\headrule}{\color{black} \headruleORIG}
\renewcommand{\headrulewidth}{1.0pt}
\usepackage{colortbl}
\arrayrulecolor{black}

\fancypagestyle{plain}{
  \fancyhead{}
  \fancyfoot{}
  \renewcommand{\headrulewidth}{0pt}
}

%%% Clear Header %%%%%%%%%%%%%%%%%%%%%%%%%%%%%%%%%%%%%%%%%%%%%%%%%%%%%%%%%%%%%%%%%%
% Clear Header Style on the Last Empty Odd pages
\makeatletter

\def\cleardoublepage{\clearpage\if@twoside \ifodd\c@page\else%
  \hbox{}%
  \thispagestyle{empty}%              % Empty header styles
  \newpage%
  \if@twocolumn\hbox{}\newpage\fi\fi\fi}

\makeatother
 
%%%%%%%%%%%%%%%%%%%%%%%%%%%%%%%%%%%%%%%%%%%%%%%%%%%%%%%%%%%%%%%%%%%%%%%%%%%%%%% 
% Prints your review date and 'Draft Version' (From Josullvn, CS, CMU)
\newcommand{\reviewtimetoday}[2]{\special{!userdict begin
    /bop-hook{gsave 20 710 translate 45 rotate 0.8 setgray
      /Times-Roman findfont 12 scalefont setfont 0 0   moveto (#1) show
      0 -12 moveto (#2) show grestore}def end}}
% You can turn on or off this option.
% \reviewtimetoday{\today}{Draft Version}
%%%%%%%%%%%%%%%%%%%%%%%%%%%%%%%%%%%%%%%%%%%%%%%%%%%%%%%%%%%%%%%%%%%%%%%%%%%%%%% 

\newenvironment{maxime}[1]
{
\vspace*{0cm}
\hfill
\begin{minipage}{0.5\textwidth}%
%\rule[0.5ex]{\textwidth}{0.1mm}\\%
\hrulefill $\:$ {\bf #1}\\
%\vspace*{-0.25cm}
\it 
}%
{%

\hrulefill
\vspace*{0.5cm}%
\end{minipage}
}

\newenvironment{bulletList}%
{ \begin{list}%
	{$\bullet$}%
	{\setlength{\labelwidth}{25pt}%
	 \setlength{\leftmargin}{30pt}%
	 \setlength{\itemsep}{\parsep}}}%
{ \end{list} }

\newtheorem{definition}{D�finition}
\renewcommand{\epsilon}{\varepsilon}

% centered page environment

\newenvironment{vcenterpage}
{\newpage\vspace*{\fill}\thispagestyle{empty}\renewcommand{\headrulewidth}{0pt}}
{\vspace*{\fill}}






\begin{document}

\maketitle


\begin{enumerate}
   
   \item In the abstract and summary, you state the goal of 'understanding the
(impact of) Fermi motion' in DVCS and beyond.  As far as I see, the variables
$x^*$ and $W^*$ you introduce and discuss are designed to 'bypass' or 'take
into account' the effect of Fermi motion for extracting DVCS on the neutron.
But that alone does not give any 'understanding' of these effects, does it?.
For this, I suppose, you would need to consider additional kinematic
quantities, and/or compare different distributions.  Can you expand a bit on
that point?\\

\textcolor{blue}{Response: When tagging, we not only reconstruct the corrected
variables $x^*$ and $W^*$, we also select a specific phase space of
the spectator. This not only impacts the result because of the Fermi
motion itself, but also because of other processes that get in our
sample because of Fermi motion (in many cases there is no more
spectator as it gets involved in the reaction). Altogether, the impact
of Fermi motion is complex and needs to be carefully modeled, this
data will offer a unique opportunity to validate theoretical
calculations. Since theory is not existent for tagged DVCS yet, we can
only speculate on the right observable to make the best test of the
calculations. However, by analogy to the tagged DIS (the main Bonus
measurement) we can assume that the angle of production of the
      spectator will be very important to study.}\\

\item You present in detail the envisaged comparison of data you would take in
the run addition with or without tagging the scattered neutron.  Can you also
say something about how you would compare that data with the one from
E12-11-003 (without the tagged proton), and what specifically one could then
learn from that comparison?  This would make the complementarity between the
run group addition and E12-11-003 more concrete.  How could one, for instance,
correct the E12-11-003 data for nuclear effects based on the measurement you
propose?\\

\textcolor{blue}{Response: Some of the answer is above already. To complete, 
      the main idea here will be to check that the integrated result from 
      E12-11-003 will not suffer from a large bias. If this is small, there is 
      not much more to do and this data will simply provide a few more points 
      for GPD fits. If the effect is significant however, the comparison to our 
      new data of the model used to correct the E12-11-003 is going to be key.  
      It would be very difficult to correct something never measured. We want 
      to emphasize here, that while tagged measurements exist for relative 
      cross sections, it remains largely unknown how this affects beam-spin 
      asymmetries.}\\

\item You emphasize on p.30 that existing model predictions for FSI effects are
not available for DVCS.  Despite this caveat, I think it might be a useful
illustration to see in a figure how the size of effects shown for inclusive
      DIS in Fig 1.5 compares with your anticipated uncertainties for $A_{LU}$ 
      in DVCS in Fig 3.12 (unless you argue that this makes little physics 
      sense).\\

\textcolor{blue}{Response: As mentioned above, there is no indication that the
beam-spin asymmetry will be affected in a comparable manner. However,
if this was the case, at large angles (~90 degrees) corrections can be
massive (about factor 2) and can definitely be seen without any
problem. In the region of interest (low momentum backward spectator
protons), cross section is modified by 10-20\%, which should be smaller
than what we can resolve since the asymmetries are only about 5\% in
      the first place.}\\

\item  Incidentally, if effects on $A_{LU}$ were similar to those in Fig 1.5, 
   then a bin of $0.12 GeV < p_{s} < 0.35 GeV$ may wash out the FSI effects for 
   $\theta_s$ = 90 degrees by averaging over regions of suppression and 
   enhancement. Perhaps you can say something about the flexibility to choose 
   different bins in $p_s$ in order to cope with such an eventuality?  (Looking 
   at Fig 3.10, I guess that you could also just select $0.12 GeV < p_s < 0.25 
   GeV$ or so, but you can probably make a more informed statement.)\\

\textcolor{blue}{Response: You are correct. The final bins haven't been chosen 
yet since they depend on the experimental resolution and acceptance of the RTPC 
for spectator protons.}\\

\item On p.16, you write that "the measurement of this channel [with tagged
spectator proton] will offer a slightly smaller amount of data as other CLAS12
measurements."  On p.27, you estimate 9 million proton-tagged neutron DVCS
events, whereas in the proposal of E12-11-003 (p.49 of the proposal to PAC 37)
I find an estimate of 25 million DVCS/BH events.  Are these the relevant
numbers to compare?\\

\textcolor{blue}{Response: Yes, you are correct. However, RG-B has been 
approved to run about half the approved days this year and BONuS12 will be 
running before the other half of RG-B. }\\

\item The right-hand plots in Figures 3.1, 3.3, and 3.4 show a rather 
   'fractured' acceptance in different azimuthal angles.  Does this have an 
   adverse effect on your acceptance in the angle phi (the one mentioned on 
   p.27)?  Do you have an acceptance plot for that angle?\\

\textcolor{blue}{Response: This is not really a problem as phi is the difference
between two planes that depend also on azimuth angles. However, the
final phi distribution, while looking continuous is full of holes when
plotted in more than 1-D. This has been a common issue in CLAS for many
studies for a long time and is resolved through acceptance
corrections. This is a tedious process, but difficult to avoid with
the detector as it has been built.}

\item Finally, let me repeat the question raised in the TAC report: can you 
   give an estimate of how much loss in beam polarization you could tolerate 
   without compromising the physics output?  Additionally, are there important
observables that may be less sensitive to this (I am thinking of the ratio of
$A_{LU}$'s in Figure 3.12)?\\

\textcolor{blue}{Response: The polarization reduction affects linearly our results.
Simply put, our error bars are constant, but we measure half the
asymmetries if we have only half polarization. This is rather critical
in this measurement because the neutron asymmetries are expected to be
much smaller than on the proton (about 5\%). If the polarization is
significantly lower than on design, it is very likely that the
comparison with E12-11-003 will not provide much significance. To test
models, the measurement is more resilient. As was mentioned, we can
select parts of the phase space where strong variations are expected
and this can be tested even with lower polarization.}\\



\end{enumerate}





\end{document}
