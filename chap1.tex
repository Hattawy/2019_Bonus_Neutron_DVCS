\chapter{Neutron Partonic Structure}
\label{chap:physics}



\section{DVCS Formalism and Observables}

A wealth of information on the structure of hadrons lies in the correlations 
between the momentum and spatial degrees of freedom of the partons. These 
correlations can be revealed through deeply virtual Compton scattering (DVCS), 
i.e., the hard exclusive lepto-production of a real photon, which provides 
access to a three-dimensional (3-D) imaging of partons within the generalized 
parton distributions (GPDs) framework \cite{Mueller:1998fv,
Ji:1996ek,
PhysRevD.55.7114,
Radyushkin:1996nd,
PhysRevD.56.5524}. The cross section for DVCS on a spin-1/2 target can be 
parametrized in terms of four helicity conserving GPDs: $H^q$, $E^q$, 
$\tilde{H}^q$, and $\tilde{E}^q$. The GPDs $H$, $E$, $\widetilde{H}$ and 
$\widetilde{E}$ are defined for each quark flavor (q = u, d, s, ... ).  
Analogous GPDs exist for the gluons, see references 
\cite{PhysRevD.56.5524,Goeke:2001tz} for further details.  In this work, we are 
mostly concerned by the valence quark region, in which the sea quarks and the 
gluons contributions do not dominate the DVCS scattering amplitude. The GPDs 
$H$ and $\widetilde{H}$ conserve the spin of the nucleon, while $E$ and 
$\widetilde{E}$ flip it \cite{Diehl:2001pm}. The $H$ and $E$ GPDs are called 
the unpolarized GPDs as they represent the sum over the different 
configurations of the quarks' helicities, whereas $\widetilde{H}$ and 
$\widetilde{E}$ are called the polarized GPDs because they are made up of the 
difference between the orientations of the quarks' helicities.

\begin{figure}
   \centering
   \includegraphics[width=0.60\textwidth,,clip,trim=0mm 20mm 0mm 0mm 
   ]{figures/DVCS2.png}
   \caption{\label{fig:dvcshandbag} Leading-twist DVCS handbag diagram with the 
   momentum definitions labeled.}
\end{figure}

The differential cross section of leptoproduction of photons for a 
longitudinally-polarized electron beam and an unpolarized nucleon target can 
be written as:
\begin{equation}
\frac{d\sigma}{dx_B\,dy\,dt\,d\phi\,d\varphi} = \frac{\alpha^3 x_B y}{16 \pi^2 
   Q^2 \sqrt{1+\epsilon^2}} \left| \frac{\mathcal{T}}{e^3} \right|^2
\end{equation}
where $\epsilon \equiv 2x_B \frac{M_n}{Q}$, $x_B=Q^2/(2p_1\cdot q_1$) is the 
Bjorken variable, $y= (p_1\cdot q_1)/(p_1\cdot k_1)$ is the photon energy 
fraction, $\phi$ is the angle between the leptonic and hadronic planes, 
$\varphi$ is the scattered electron's azimuthal angle, $Q^2= -q_1^2$, and 
$q_1=k_1-k_2$. The particle momentum definitions are shown in 
Figure~\ref{fig:dvcshandbag}. The momentum transfer where the nucleon is 
initially at rest, $\Delta = p_1-p_2$ and $t=\Delta^2$. The Bjorken variable  
is related to another scaling variable called skewedness:
\begin{equation}
\xi = \frac{x_B}{2 - x_B} + \mathcal{O}(1/Q^2).
\end{equation}

The amplitude is the sum of the DVCS, the Bethe-Heitler (BH), and the 
interference amplitudes, and when squared has terms
\begin{equation}
   \mathcal{T}^2 = \left|\mathcal{T}_{\text{BH}}\right|^2 + 
   \left|\mathcal{T}_{\text{DVCS}}\right|^2 + \mathcal{I}
\end{equation}
where the first is the BH contribution, the second is the DVCS part, and the last 
term is the interference part,
\begin{equation}
   \mathcal{I} = \mathcal{T}_{\text{DVCS}}\mathcal{T}_{\text{BH}}^{*} + 
   \mathcal{T}_{\text{DVCS}}^{*}\mathcal{T}_{\text{BH}}.
\end{equation}
The corresponding amplitudes are calculated with the diagrams shown in Figures 
\ref{fig:dvcshandbag} and \ref{fig:BHhandbag}. The details of contracting the 
DVCS tensor with various currents and tensors can be found 
in~\cite{Belitsky:2001ns}.
\begin{figure}[!hbt]
   \centering
   \includegraphics[width=0.65\textwidth]{figures/BH.png}
   \caption{\label{fig:BHhandbag} BH handbag diagrams.}
\end{figure}
%
The resulting expressions for the amplitudes are
\begin{align}
   \left|\mathcal{T}_{\text{BH}}\right|^2 &= 
   \frac{e^6(1+\epsilon^2)^{-2}}{x_B^2\,y^2\,t\,
   \mathcal{P}_1(\phi)\mathcal{P}_2(\phi)} \left\{ c_0^{\text{BH}} + 
   \sum_{n=1}^{2}\left[ c_n^{\text{BH}}\cos(n\phi) +s_n^{\text{BH}}\sin(n\phi) 
   \right] \right\} \\
\left|\mathcal{T}_{\text{DVCS}}\right|^2 &= \frac{e^6}{y^2\,Q^2}\left\{ 
c_0^{\text{DVCS}} + \sum_{n=1}^{2}\left[ c_n^{\text{DVCS}}\cos(n\phi) 
   +s_n^{\text{DVCS}}\sin(n\phi) \right] \right\}\\
   \mathcal{I} &= \frac{e^6(1+\epsilon^2)^{-2}}{x_B\,y^3\,t\,
   \mathcal{P}_1(\phi)\mathcal{P}_2(\phi)}\left\{ c_0^{\mathcal{I}} + 
   \sum_{n=1}^{3}\left[ c_n^{\mathcal{I}}\cos(n\phi) 
   +s_n^{\mathcal{I}}\sin(n\phi) \right] \right\}
	\label{eq:sin}
\end{align}
%
The functions $c_0$, $c_n$, and $s_n$ are called \emph{Fourier coefficients}.  
They depend on the kinematic variables and the operator decomposition of the 
DVCS tensor for a target with a given spin. At leading twist there is a 
straightforward form factor decomposition which relates the vector and 
axial-vector operators with the so-called Compton form factors 
(CFFs)~\cite{Belitsky:2000gz}. The Compton form factors appearing in the DVCS 
amplitudes are integrals of the type
%
%
\begin{equation}
   \mathcal{F} = \int_{-1}^{1} dx F(\mp x,\xi,t) C^{\pm}(x,\xi)
\end{equation}
where the coefficient functions at leading order take the form
\begin{equation}
   C^{\pm}(x,\xi) = \frac{1}{x-\xi + i\epsilon} \pm \frac{1}{x+\xi - 
   i\epsilon}.
\end{equation}
%
We plan on measuring the beam spin asymmetry as a function of $\phi$
\begin{equation}
   A_{LU}(\phi) = \frac{d\sigma^{\uparrow}(\phi) - 
   d\sigma^{\downarrow}(\phi)}{d\sigma^{\uparrow}(\phi) + 
   d\sigma^{\downarrow}(\phi)}
\end{equation}
%
where the arrows indicate the electron beam helicity. 

\section{Neutron GPDs}
The measurement of free proton DVCS has been the focus of a worldwide effort 
\cite{PhysRevLett.87.182002,
   PhysRevLett.87.182001,
   PhysRevD.75.011103,
   Girod:2007aa,
   PhysRevC.92.055202,
   PhysRevLett.99.242501,
   PhysRevC.80.035206,
   PhysRevLett.114.032001,
   Jo:2015ema}
involving several accelerator facilities such as Jefferson Lab, DESY and  
CERN. These measurements now enable the extractions of GPDs and a 3-D 
tomography of the free proton \cite{Guidal:2013rya, PhysRevD.95.011501}. The 
aim of this proposal is to enhance the neutron GPD measurements along the 
approved CLAS12 experiment E12-11-003~\cite{neutronDVCS}, which will also 
measure the quasi-free neutron DVCS by detecting the scattered neutron in 
deuterium.  

In the fits of PDFs, for example \cite{Ball:2014uwa} , neutrino and deuterium 
data allow to make a flavor dependent extraction, this option is not available 
for GPDs yet due to the lack of reliable data. Indeed, the observables of DVCS, 
such as cross sections and beam-spin asymmetries are much smaller on neutron 
targets, while the nuclear effects in deuterium increase uncertainties. These 
issues have lead to results \cite{Mazouz:2007aa} which are not precise enough 
to help in a flavor dependent GPD extraction. To achieve this performance, one 
will need a large quantity of high precision data, a goal set by E-12-11-003.  
However, the impact of the uncertain initial state and the final state 
interactions on the integration of these data in global fits remain unclear.  
With this proposal, we propose to both provide more data on the neutron,
which is always important, but most importantly these data will have completely 
independent systematic uncertainties from the E-12-11-003 data. That will make 
them complementary and indicate in what ways these methods are equivalent or 
potentially need corrections.

Neutron DVCS is also hoped to provide an important contribution to the 
extraction of the GPD $E$~\cite{dHose:2016mda}. The reasoning behind this 
expectation is as follows: the GPD $E$ never appears to be dominant in the 
usually measured DVCS observables, so as a sub-leading contribution it is 
always affected with large error bars. Actually, recent extractions 
\cite{Dupre:2017hfs,Moutarde:2018kwr} show that we still barely have any 
constraint on $E$ using all the world proton data. However, as form factors 
often appear in the expressions of DVCS observables because of the Bethe 
Heitler process (see below) the situation is very different for protons and 
neutrons.  Indeed, the $F_1$ form factor of the neutron is very small, making 
the $E$ GPD more prominent in some of the neutron DVCS observables. Most 
notably the $\sin$ component of the beam spin asymmetry is defined as:

\begin{equation}
   A_{LU}^{\sin\phi} = \frac{1}{\pi} \int_{\pi}^{\pi} d\phi \sin\phi 
   A_{LU}(\phi)
\end{equation}
is proportional to the following combination of Compton form 
factors~\cite{Guidal:2013rya}

\begin{equation}
A_{LU}^{\sin\phi} \propto \Im m \left [ F_1 \mathcal{H} + \xi(F_1 + F_2) \mathcal{\tilde H} 
   - \frac{t}{4M^2} F_2  \mathcal{E} \right ],
\end{equation}
highlighting the effect of a suppressed $F_1$ on the main term.
This lead to the idea that neutron DVCS will also help significantly to constrain the 
GPD $E$. This goal is of course driven by the long standing objective of GPD physics to
measure the Ji sum rule in the nucleon:

\begin{equation}
J_q \,=\, {1 \over 2} \, \int_{-1}^{+1} d x \, x \, 
\left[ H^{q}(x,\xi,t = 0) + E^{q}(x,\xi,t = 0) \right],
\label{eq:dvcs_spin}
\end{equation}
which links the total angular momentum ($J_q$) carried by each quark $q$ to the 
sum of the second moments over $x$ of the GPDs $H$ and $E$, that will complete 
the decomposition of the nucleon spin in its various components 
\cite{Ji:1996ek,Leader:2013jra}. 


\section{Measuring the neutron DVCS}

\subsection{Two new methods for two objectives}

As shown in Fig.~\ref{fig:ndvcsexclusive}, one can measure all final state 
particles of the DVCS reaction on deuterium using CLAS12 and Bonus12. This 
method, while perfect on paper, leads to an efficiency problem
as both the measurement of the spectator proton and of the neutrons are challenging
and have low efficiency (intrinsically for the neutron detection and by the 
limitation of phase space available for the proton). However, it is possible to 
ensure the exclusivity of a process when missing one of the final state 
particles by applying cuts on missing mass, momentum and energy. This strategy 
can be used, either to leave the spectator proton or the neutron undetected.  
The former approach is being used in the ongoing E12-11-003. We propose here to 
add the other option, detecting the spectator proton but not the neutron, in 
order to increase the amount of available data, but most importantly, to 
confirm the equivalency of both methods. Moreover, we will be able to
perform the fully exclusive measurement, but with rather limited statistics. We
intend to use this over-constrained last measurement to study the systematic 
effects linked to Fermi motion and final state interaction effects on our DVCS 
observable of interest, i.e., beam-spin asymmetry.

\begin{figure}
   \centering
   \includegraphics[width=0.60\textwidth,,clip,trim=0mm 0mm 0mm 0mm 
   ]{figures/dvcs_feynman-figure1.pdf}
   \caption{\label{fig:ndvcsexclusive} Fully exclusive neutron DVCS diagram in 
   deuterium. }
\end{figure}

\subsection{Proton-Tagged Neutron DVCS with BONuS12}

The proton-tagged neutron detection scheme is described in 
Fig.~\ref{fig:ndvcstagged}, where we see
that a detector for low energy proton spectators is necessary. Here, we propose to use the 
Bonus radial TPC, which is designed to make a similar type of measurement for inclusive 
deeply inelastic scattering. The first goal of the present proposed experiment is simply
to provide more data in the field of neutron GPD. Indeed, the measurement of neutron
DVCS is very challenging and very little published data are available at this 
point. 

\begin{figure}
   \centering
   \includegraphics[width=0.60\textwidth,,clip,trim=0mm 0mm 0mm 0mm 
   ]{figures/dvcs_feynman-figure0.pdf}
   \caption{\label{fig:ndvcstagged} Proton-tagged neutron DVCS diagram in 
   deuterium.  }
\end{figure}

Second, we observe that the systematics from this measurement are going to be 
mostly independent of the ones from E12-11-003. Indeed, while this measurement 
will be missing one of the high energy product of the reaction leading to more 
uncertainty in exclusivity cuts, it will detect the spectator proton, 
significantly helping in the understanding of the nuclear effects. This has 
interestingly an impact on both initial state and final state effects. In the 
initial state, the neutron is in fact not at rest and carries some Fermi 
momentum, which can be directly inferred from the kinematic of the spectator 
proton, thanks to the simple two body deuterium.  On the final state side, 
detecting the spectator proton in a certain range of momentum and angle allows 
to significantly reduce the probability of final state interactions to have 
occurred.

In order to resolve the initial state issue, we evaluate the standard Lorentz 
invariant $x$ and the $\gamma^{*}n$ invariant mass ($W$) in terms of the 
spectator kinematics. Both invariants, $x$ and $W$, acquire a star to indicate  
true invariants rather than the values calculated assuming a stationary, 
on-shell target:
\begin{equation}
   x^* = \frac{Q^2}{2M_{N}Ey (2-\alpha_{sp})} = \frac{x_B}{2-\alpha_{sp}},
\end{equation}
\begin{equation}
   W^* = M^{*2} - Q^2+ 2MEy(2-\alpha_{sp}),
\end{equation}

where $\alpha_{sp} = \frac{E_{s} - p^{z}_{s}}{M_N}$, with $M_N$, $E_{s}$, and 
$p^{z}_{s}$ are the on-shell mass, energy, and z-momentum component of the 
spectator proton. The off-shell mass of the bound nucleon is given by $M^{*2} = 
(M_D - E_{s})^{2} - p^{2}_{s}$, where $M_D$ is the rest mass of the deuterium.

To understand the regions where final state interactions are expected to be 
significant, we look at the spectator momentum and angle relative to the 
momentum transfer, $\theta_{s}$.  In 
Figure~\ref{fig:deuteronFSI}~\cite{CiofidegliAtti:2003pb,CiofidegliAtti:2002as}, 
we see calculations for the inclusive case.  At low recoil momentum and 
backwards spectator angle, the FSIs are negligible, whereas at high momenta 
perpendicular to the momentum transfer, FSIs are maximized.

\begin{figure}
   \centering
   \includegraphics[width=0.9\textwidth]{figures/edit_FSI_quasielastic_Atti_2003.pdf}
   \caption{\label{fig:deuteronFSI} Ratio of cross sections for the FSI model 
   from~\cite{CiofidegliAtti:2003pb} to Plane Wave Impulse Approximation (PWIA) 
   calculation as a function of the spectator momentum (left) and spectator 
   angle (right). The caption on the left figure has been edited from the 
   original paper to correct for a typo.}
\end{figure}

It is important to note that the detection of the neutron or the spectator 
proton are not equivalent. While it could appear to be so after applying the 
exclusivity cut, it is not the case because of the large uncertainty (a percent 
or more) in the energy measurement of photons. This uncertainty is larger
than the momentum of the spectator proton and therefore completely hinders our 
capability to reconstruct it from missing momentum and energy methods.

One should mention that our recent 
measurement~\cite{Hattawy:2018liu} of incoherent DVCS on helium conducted during the 6 
GeV era (E08-024) have shown significant modification of the proton beam 
spin asymmetry in $^4$He. These results are shown in 
Figure~\ref{fig:incoh_EMC_ratio_ALU_proton}, where we observe a much smaller asymmetry for 
bound protons. However, to this day, it is not possible to say if this suppression should
be associated to the parasitic
nuclear effects presented above or to a more fundamental source linked to the EMC effect. 
While not directly comparable, this unexpectedly large deviation seems to hint to large
nuclear effects in incoherent nuclear DVCS that have not been explored at all on the theory 
side to this day.

\begin{figure}[tb]
\centering
\includegraphics[width=9.8cm]{figures/ALU_ratioInc_t_shortscenrario-without-error-onX.pdf}
\caption{ The $A_{LU}$ ratio of the bound to the free proton at 
   $\phi$~=~90$^{\circ}$ as a function of $t$. The black squares are from 
   CLAS-eg6 experiment~\cite{Hattawy:2018liu}, the green circle is the HERMES 
   measurement \cite{Airapetian:2009cga}.  The error bars represent the 
   statistical uncertainties, while the gray band represents the systematic 
   uncertainties. The blue and red curves are results of off-shell calculations 
   \cite{Liuti:2005gi}. The solid and dashed black curves are from on-shell 
   calculations \cite{Guzey:2008fe}.} \label{fig:incoh_EMC_ratio_ALU_proton}
\end{figure}

In conclusion, the measurement of this channel will offer a slightly smaller amount of 
data as other CLAS12 measurements. However, the key lies in the different and 
independent systematic effects of this new measurement method. In 
particular, we will study and reduce the impact of nuclear effects, which will 
significantly help in the interpretation of E12-11-003 
data~\cite{neutronDVCS}. 

\subsection{Fully exclusive neutron DVCS with BONuS12}

As highlighted above, the fully exclusive measurement of neutron DVCS is 
difficult since two hard-to-detect particles have to be measured, a slow proton 
and a fast neutron.  However, it can still provide relevant information in a 
limited kinematic space. As this measurement is fully exclusive, it provides 
over-constraints by using missing mass, momentum and energy cuts. This will 
give us a particularly clean sample of data with a minimal amount of 
corrections to be applied and the best control over systematic errors. We 
propose to use this sample, that can only be obtained using a recoil detector, 
to study the effects of Fermi motion, final state interactions and incomplete 
detection of the final state described above. This will confirm assumptions 
made in other neutron DVCS measurements and help understand better their 
systematic errors.



