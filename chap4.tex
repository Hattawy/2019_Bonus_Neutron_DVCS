 \chapter{Additional Physics Measurements with Run Group F}
 \label{chap:additional}

The combination of the high luminosity available at Jefferson Lab, the large 
acceptance of CLAS12 detector and the BONuS12 RTPC offers an amazing 
opportunity to advance our understanding of long standing mysteries in QCD.  
This new development in detection capabilities, will allow the study of medium 
modification with a handle on Fermi motion uncertainties and FSI effects. It is 
therefore clear that the focus of the neutron DVCS proposal is only a fraction 
of the physics that can be achieved by successfully analyzing the Run Group F 
data with highly polarized beam. This data will be a gold mine, which will 
allow us to investigate in a unique way several important physics questions and 
conquer new territories in the nuclear QCD land. Some of the topics of interest 
to increase the physics outcome of BONuS12 polrized beam data are:

\begin{itemize}
\item Coherent DVCS and deeply virtual mesons production (DVMP) off deuteron.  
   For DVMP, we can study for example $\pi^0$, $\phi$, $\omega$ and $\rho$ 
      mesons.

\item Incoherent proton DVCS and DVMP off bound deuterium.

\item Deep virtual $\pi^0$ production off neutron, which is interesting by 
   itself but also the background of DVCS measurements.
\item Semi-inclusive reaction p(e,e'p)X to study the $\pi^0$ cloud of the 
   proton and $D(e, e' pp_S)X$ to study the $\pi^-$ cloud of the neutron, at 
      very low proton momenta.
\item The medium modification of the transverse momentum dependent parton 
   distributions.
\end{itemize}

