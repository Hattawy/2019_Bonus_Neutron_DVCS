\chapter*{Introduction\markboth{\bf Introduction}{}}
\label{chap:intro}
\addcontentsline{toc}{chapter}{Introduction}

Inclusive deep inelastic scattering (DIS) experiments have been instrumental in 
advancing our understanding of the QCD structure of nucleons in the past. More
recently, hard exclusive experiments such as Deeply Virtual Compton Scattering (DVCS) and 
Deeply Virtual Meson Production (DVMP) provide an important new probe that 
allows us to describe both the motion of partons and their transverse 
spatial structure in nucleons through the generalized parton distribution (GPD) framework.  
The GPDs correspond to the coherence between quantum 
states of different (or same) helicity, longitudinal momentum, and transverse 
position. In an impact parameter space, they can be interpreted as a 
distribution in the transverse plane of partons carrying a certain longitudinal 
momentum~\cite{Burkardt-2000,Diehl-2002,Belitsky-2002}. A crucial feature of 
GPDs is the access to the transverse position of partons which, combined with 
their longitudinal momentum, leads to the total angular momentum of 
partons~\cite{Burkardt-2005}. This information is not accessible to inclusive 
DIS which measures probability amplitudes in the longitudinal plane. \\
Inclusive deep inelastic scattering (DIS) experiments have been instrumental in 
advancing our understanding of the QCD structure of nucleons in the past. More
recently, hard exclusive experiments such as Deeply Virtual Compton Scattering (DVCS) and 
Deeply Virtual Meson Production (DVMP) provide an important new probe that 
allows us to describe both the motion of partons and their transverse 
spatial structure in nucleons through the generalized parton distribution (GPD) framework.  
The GPDs correspond to the coherence between quantum 
states of different (or same) helicity, longitudinal momentum, and transverse 
position. In an impact parameter space, they can be interpreted as a 
distribution in the transverse plane of partons carrying a certain longitudinal 
momentum~\cite{Burkardt-2000,Diehl-2002,Belitsky-2002}. A crucial feature of 
GPDs is the access to the transverse position of partons which, combined with 
their longitudinal momentum, leads to the total angular momentum of 
partons~\cite{Burkardt-2005}. This information is not accessible to inclusive 
DIS which measures probability amplitudes in the longitudinal plane. \\


A high luminosity facility such as Jefferson Lab offers a unique opportunity to 
map out the three-dimensional quark and gluon structure of nucleons and nuclei.  
While most of submitted proposals to JLab Program Advisory Committee (PAC) have 
focused on the studies of the 3D proton structure considered as one of the main 
motivations for the JLab 12 GeV upgrade, we propose here to extend the 
measurements to quasi-free neutrons in deuterium with BONuS12 experiment. Such
measurement is critical in the Jefferson Lab GPD program for two aspects, first
as neutrons offer the only access to flavor decomposition for GPDs, second because
of the importance of the $E$ GPD to measure the Ji sum rule~\cite{} (add Ji's and a review).
(RD Maybe add more details about that here)

Recent results from CLAS on incoherent DVCS from a 4He target have shown...
(RD Complete paragraph to explain why we should be worried about FSI) 

In consequence, it is critical to understand better how the initial state and the 
break-up of the deuterium might affect BSA. The run group F is the perfect occasion
to gather data on this topic, which can be done with the simple addition of beam 
polarization to the setting of the run group!

