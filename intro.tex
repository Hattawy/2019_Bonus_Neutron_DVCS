\chapter*{Introduction\markboth{\bf Introduction}{}}
\label{chap:intro}
\addcontentsline{toc}{chapter}{Introduction}

Inclusive deep inelastic scattering (DIS) experiments have been instrumental in 
advancing our understanding of the QCD structure of nucleons in the past. More
recently, hard exclusive experiments such as Deeply Virtual Compton Scattering (DVCS) and 
Deeply Virtual Meson Production (DVMP) have provided important new probes that 
allows us to explore both the longitudinal motion of partons and their 
transverse spatial structure in nucleons through the generalized parton 
distribution (GPD) framework.  The GPDs correspond to the coherence between 
quantum states of different (or same) helicity, longitudinal momentum, and 
transverse position.  In an impact parameter space, they can be interpreted as 
a distribution in the transverse plane of partons carrying a certain 
longitudinal momentum~\cite{Burkardt-2000,Diehl-2002,Belitsky-2002}. A crucial 
feature of GPDs is the access to the transverse position of partons which, 
combined with their longitudinal momentum, leads to the total angular momentum 
of partons~\cite{Burkardt-2005}. This information is not accessible to 
inclusive DIS which measures probability amplitudes in the longitudinal plane.  

A high luminosity facility such as Jefferson Lab offers a unique opportunity to 
map out the three-dimensional quark and gluon structure of nucleons and nuclei.
This is one of the flagship of the JLab 12 GeV scientific program, such that many
approved proposals to the JLab Program Advisory Committee (PAC) have 
focused on studies of the 3D proton or neutron structure. We propose here to 
extend the DVCS measurements of neutrons to the effective quasi-free neutrons 
obtained using the BONuS12 
experiment. Measurements on the neutron are critical in the Jefferson Lab GPD 
program for two reasons, first as neutrons offer the only access to flavor 
decomposition for GPDs, second because of the importance of the $E$ GPD to 
measure the Ji sum rule~\cite{Ji:1996ek,Leader:2013jra}.

Recent results from CLAS on incoherent DVCS from a $^4$He target have shown that
even in light nuclei, large nuclear effects can be observed~\cite{Hattawy:2018liu}.
As this recent measurement shows a deviation of $\approx 30$\% for the beam spin asymmetry (BSA)
of DVCS of bound protons in helium, one can assume that a sizable correction 
will be necessary in the neutron case as well; although reduced in the much 
less tightly bound deuterium. This could be particularly dramatic as the BSA 
expected on neutrons are rather small, making the nuclear suppression observed in helium
of a similar size as the signal expected for the neutron. We propose here to transpose the methods 
developed for BONuS12 for PDF measurements~\cite{bonus12} to the GPDs. In 
particular, we are looking how the initial state and the break-up of the 
deuterium might affect the measured BSA. The upcoming Run group F (solely composed of
the BONuS12 experiment) scheduled 
for February-April 2020, is the perfect occasion to gather data on this topic, 
which can be done with the simple addition of beam polarization to the setting 
of the run group!

